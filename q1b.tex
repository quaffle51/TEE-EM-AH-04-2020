% !TeX root = ./TMA04.tex
Let $w(x)=1$ and $\alpha=\beta=\gamma=1$, so that \eqref{eq:1.2} becomes
\begin{equation}
\label{eq:1.3}
\deriv{^2y}{x^2} + \lambda y =0,\quad y(0) = 0,\;\;(1-\lambda )y(1) + y^\prime(1)=0.
\end{equation}
Then to  find the non-trivial  stationary paths, the eigenfunctions of $y$ (normalised so that $C[y]=1$) and the values of the Lagrange multipliers, consider the following.

There are three cases to consider regarding $\lambda$, namely $\lambda=0$, $\lambda<0$, and $\lambda>0$. So, considering each case in turn as follows.
\begin{description}[itemindent=-0.9cm]
\begin{comment}
========================================================================================================================================================
\end{comment}
\item[if $\lambda = 0$]
set $\lambda = 0$ in \eqref{eq:1.3}:
\begin{align*}
	\deriv{^2y}{x^2}&=0,\quad y(0) = 0,\;\;y(1) + y^\prime(1)=0.\\
	\deriv{y}{x} &= A,\qtq{where $A$ is an arbitrary constant.}\\
	y&= Ax + B,\qtq{where $B$ is also an arbitrary constant.}
\end{align*}
Applying the boundary conditions to $y^\prime$ and $y$ gives $A=0$ and $B=0$ and therefore $y(x)=0$. This is a trivial solution.
\begin{comment}
========================================================================================================================================================
\end{comment}
\item[If $\lambda < 0$]
set $\lambda = -\mu^2\;\;(\mu > 0)$ in \eqref{eq:1.3}:
\begin{align*}
	\deriv{^2y}{x^2} - \mu^2  y =0,\quad y(0) = 0,\;\;(1- \mu^2)y(1) + y^\prime(1)=0.
\end{align*}
The auxiliary equation is 
\begin{align*}
	&r^2 - \mu^2 = 0, \qtq{with} r_1 = -\mu,\;\; r_2 = \mu.\\
	&y(x) = A\e^{-\mu x} + B\e^{\mu x}.
\end{align*}
Applying the first boundary conditions to $y(x)$ with $x=0$ and $y(o)=0$ gives
\[
	0=A+B\qtq{therefore} A=-B,
\]
and so,
\[
	y(x) = -B\e^{-\mu x} + B\e^{\mu x}.
\]
Differentiating $y$,
\begin{align*}
	y^\prime(x) &= B\mu\e^{-\mu x}+ B\mu\e^{\mu x},
\end{align*}
Applying the second boundary condition to determine $B$,
\begin{align*}
(1-\mu^2) y(1) + y^\prime(1) &= (1-\mu^2) \lr{-B\e^{-\mu} + B\e^{\mu}} + B\mu\e^{-\mu}+ B\mu\e^{\mu} = 0,\\
                          &= B\lrc{(1-\mu^2) \lr{-\e^{-\mu} + \e^{\mu}} + \mu\e^{-\mu}+ \mu\e^{\mu}} =0.
\end{align*}
By assumption $\mu \ne 0$, so the expression immediately above can only be satisfied if $B=0$  and therefore y(x) = 0.  This also is a trivial solution.
\begin{comment}
========================================================================================================================================================
\end{comment}
\item[If $\lambda > 0$]
set $λ = \mu^2\;\;(\mu > 0)$ in \eqref{eq:1.3}:
\begin{align*}
	\deriv{^2y}{x^2} + \mu^2 y =0,\quad y(0) = 0,\;\;(1-\mu^2)y(1) + y^\prime(1)=0.
\end{align*}
The auxiliary equation is
\begin{align*}
	&r^2 + \mu^2 = 0, \qtq{with} r_1 = -i\mu,\;\; r_2 = i\mu.
\end{align*}
and the general solution is given by
\[
	y(x) = A \sin \mu x + B\cos \mu x.
\]
Applying the first boundary condition $y(0) = 0$, gives,
\[
	y(0) = A \cancelto{0}{\sin 0}  + B\cancelto{1}{\cos 0},\qtq{and therefore, $B=0$.} 
\]
Thus, the updated solution is, \[y(x) = A\sin \mu x.\]

Differentiating $y(x)$ gives,
\[
	y^\prime (x) = A\mu\cos \mu x.
\]
Applying the second boundary conditions $(1-\mu^2)y(1) + y^\prime(1)=0$:
\begin{align}
	&(1-\mu^2)A\sin \mu + A\;\mu\cos \mu = 0,\nonumber\\
	&\text{and factoring out $A$ and assuming $A\ne0$, gives,}\nonumber\\
	\label{eq:1.4}
	&(1-\mu^2)\sin \mu + \;\mu\cos \mu = 0,\\
	&\frac{(1-\mu^2)}{\mu} = -\frac{\cos \mu}{\sin \mu},\nonumber\\
	\label{eq:1.5}
	&\frac{(\mu^2-1)}{\mu} = \cot \mu.
\end{align}
To solve \eqref{eq:1.5} for $\mu$ a graph of both sides of \eqref{eq:1.5} can be used as shown in Figure~\ref{fig:1}.
From the graphs shown in Figure~\ref{fig:1} $\mu_1\approx 1.208$, $\mu_2\approx 3.448$, $\mu_3\approx 6.441$, $\mu_4\approx 9.530$, $\mu_5\approx 12.646$ and it is seen that as $n$ increases $\mu_n$ approaches $(n-1)\pi$, $n=1,2,\ldots$. Thus, the eigenvalues $\lambda_n$ approach $(n-1)^2\pi^2$ as $n$ increases. Hence, the eigenfunctions corresponding to the eigenvalues $\lambda_n$ are, 
\begin{equation}
\label{eq:1.6}
	y_n(x) = A_n \sin\lr{\sqrt{\lambda_n}x},\qquad n = 1,2,\ldots.
\end{equation}
To normalise the eigenfunctions \eqref{eq:1.6} consider the following.
\begin{figure}[h!]
\centering{
\resizebox{\textwidth}{!}{\import{/home/gordon/OU/TMA04/images/}{desmos-graph.pdf_tex}}
\caption{Graphs of $y=\lr{\dfrac{\mu^2-1}{\mu}}$ and $y=\cot(\mu)$.}
\label{fig:1}
}
\end{figure}

The inner product (with unit weight function $w$) where $(y,y)_w=1$ will be used to normalise the expression for $y$ given in \eqref{eq:1.6},\marginnote{See HB p31.}
\begin{equation}
\label{eq:1.7}
	(y,y)_w = \Int{a}{b}{x}y(x)^2 = 1,
\end{equation}
by determining the value of $A_n$ as follows.  

After substituting into \eqref{eq:1.7} for $y_n(x)$ from \eqref{eq:1.6} the following integral is obtained.
\begin{equation*}
\label{eq:1.8}
	\Int{0}{1}{x} A_n^2 \sin^2 \lr{ \sqrt{\lambda_n}x }= 1,
\end{equation*}
\[
	A_n^2 \Int{0}{1}{x} \sin^2 \lr{ \sqrt{\lambda_n}x }= 1.
\]
Using the trig identity $\sin^2\alpha=\half\lr{1-\cos2\alpha}$\marginnote{See HB p38 for trig identities.} gives,
\[
	\frac{A_n^2}{2} \Int{0}{1}{x}  \lr{1-\cos\lr{2\sqrt{\lambda_n}x}}= 1.
\]
\[
	\frac{A_n^2}{2} \bigg[x - \frac{1}{2\sqrt{\lambda_n}}\sin\lr{2\sqrt{\lambda_n}x}\bigg]_0^1= 1.
\]
\[
	\frac{A_n^2}{2} \lr{1 - \frac{1}{2\sqrt{\lambda_n}}\sin\lr{2\sqrt{\lambda_n}}}= 1.
\]
\[
	\frac{A_n^2}{4\sqrt{\lambda_n} } \lr{2\sqrt{\lambda_n} - \sin\lr{2\sqrt{\lambda_n}}}= 1.
\]
Using the trig identity $\sin 2\alpha = 2\sin\alpha\cos\alpha$ then,
\[
	\frac{A_n^2}{4\sqrt{\lambda_n} } \lr{2\sqrt{\lambda_n} - 2\sin\lr{\sqrt{\lambda_n}}\cos\lr{\sqrt{\lambda_n}}}= 1.
\]
From \eqref{eq:1.4}
\[
	-\sin\lr{\sqrt{\lambda_n}} = \frac{\sqrt{\lambda_n}}{1-\lambda_n}\cos\lr{\sqrt{\lambda_n}},
\]
and so,
\[
	\frac{A_n^2}{\cancelto{2}{4\sqrt{\lambda_n}} }\;\;\lr{\cancelto{1}{2\sqrt{\lambda_n}} + \frac{\cancelto{1}{2\sqrt{\lambda_n}}}{1-\lambda_n }\cos\lr{\sqrt{\lambda_n}}\cos\lr{\sqrt{\lambda_n}}}= 1.
\]
\[
	\frac{A_n^2}{2} \lr{1+ \frac{1}{1-\lambda_n }\cos^2\lr{\sqrt{\lambda_n}}}= 1.
\]
\[
	\frac{A_n^2}{2\lr{1-\lambda_n }} \lr{\lr{1-\lambda_n } + \cos^2\lr{\sqrt{\lambda_n}}}= 1.
\]
\[
	A_n^2 = \frac{2\lr{1-\lambda_n }}{{\lr{1-\lambda_n } + \cos^2\lr{\sqrt{\lambda_n}}}}.
\]
Therefore,
\[
	A_n = \sqrt{\frac{2\lr{1-\lambda_n }}{\lr{1-\lambda_n } + \cos^2\lr{\sqrt{\lambda_n}}}},
\]
and the normalised eigenfunctions are given by,
\[
	y_n(x) = \sqrt{\frac{2\lr{1-\lambda_n }}{\lr{1-\lambda_n } + \cos^2\lr{\sqrt{\lambda_n}}}}\;\; \sin\lr{\sqrt{\lambda_n}x}.
\]
\begin{comment}
As $n\rightarrow\infty$, $A_n\rightarrow\sqrt{2}$ and so,
\[
	y_n(x) = \sqrt{2} \sin\lr{\sqrt{\lambda_n}x},\qtq{as} n\rightarrow\infty.
\]
\end{comment}
\end{description}

