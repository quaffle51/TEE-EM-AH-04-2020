% !TeX root = ./TMA04.tex
The Rund-Trautman Identity:
\begin{align}
\label{5.5}
	\pderiv{L}{x}\xi +  \boxed{p\dot{\xi}} + \pderiv{L}{t}\tau - \boxed{H\dot{\tau}}=0,\\
\label{5.6}
	-\pderiv{L}{x}\xi - \pderiv{L}{t}\tau = \boxed{p\dot{\xi}}  - \boxed{H\dot{\tau}}
\end{align}
The RHS:
\begin{align}
\label{5.7}
	\deriv{}{t}\lrs{p\xi-H\tau} &= \boxed{p\dot{\xi}} +\dot{p}\xi - \boxed{H\dot{\tau}} - \dot{H}\tau\\
\label{5.8}
	&= \boxed{p\dot{\xi}} - \boxed{H\dot{\tau}} - \dot{H}\tau +\dot{p}\xi 
\end{align}
\eqref{5.6} into \eqref{5.8}:
\begin{align}
\label{5.9}
	\deriv{}{t}\lrs{p\xi-H\tau}&= -\pderiv{L}{x}\xi - \pderiv{L}{t}\tau  - \dot{H}\tau +\dot{p}\xi 
\end{align}
collect terms in $\xi$
\begin{align}
\label{5.10}
	\deriv{}{t}\lrs{p\xi-H\tau}&=  \xi\lr{\dot{p} -\pderiv{L}{x}} - \pderiv{L}{t}\tau  - \dot{H}\tau
\end{align}
From the question, the \textit{Hamiltonian} is defined as
\begin{align}
\label{5.11}
	H &= p\dot{x} - L\qtq{and its derivative is given by}\\
\label{5.12}
	\deriv{H	}{t} = \dot{H} &= \deriv{}{t}\lr{p\dot{x}}-\deriv{L}{t}\\
\label{5.13}
	&= p\ddot{x} +\dot{p}\dot{x} -\deriv{L}{t}
\end{align}
As shown in part~(b)
\begin{align}
\label{5.14}
\deriv{L}{t} &= \pderiv{L}{t}+\pderiv{L}{x}\dot{x} + p\ddot{x}
\end{align}
\eqref{5.14} into \eqref{5.13}
\begin{align}
\label{5.15}
	\dot{H} &= p\ddot{x} +\dot{p}\dot{x} -\lr{\pderiv{L}{t}+\pderiv{L}{x}\dot{x} + p\ddot{x}}\\
\label{5.16}
	&= p\ddot{x} +\dot{p}\dot{x} -\pderiv{L}{t}-\pderiv{L}{x}\dot{x} - p\ddot{x}\\
\label{5.17}
	\dot{H} + \pderiv{L}{t} &= p\ddot{x} +\dot{p}\dot{x} -\pderiv{L}{x}\dot{x} - p\ddot{x}
\end{align}
Multiply \eqref{5.17} by $-\tau$ and cancel like terms
\begin{align}
\label{5.18}
	-\dot{H}\tau - \pderiv{L}{t}\tau &= \cancelto{1}{-p\ddot{x}\tau} -\dot{p}\dot{x}\tau + \pderiv{L}{x}\dot{x}\tau \cancelto{1}{+ p\ddot{x}\tau}\\
\label{5.19}
	&=  -\dot{p}\dot{x}\tau + \pderiv{L}{x}\dot{x}\tau\\
\label{5.20}
	&= - \lr{\dot{p} - \pderiv{L}{x}}\dot{x}\tau
\end{align}
Substitute \eqref{5.20} into \eqref{5.10}
\begin{align}
\label{5.21}
	\deriv{}{t}\lrs{p\xi-H\tau}&=  \xi\lr{\dot{p} -\pderiv{L}{x}} - \lr{\dot{p} - \pderiv{L}{x}}\dot{x}\tau
\end{align}
Finally,
\begin{align}
\label{5.22}
	\deriv{}{t}\lrs{p\xi-H\tau}&=  \lr{\xi- \dot{x}\tau}\lr{\dot{p} -\pderiv{L}{x}} 
\end{align}
as required.

