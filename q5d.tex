% !TeX root = ./TMA04.tex
The \textit{Rund-Trautman identity} is given as
\begin{equation}
	\label{eq:5.3}
	\pderiv{L}{x}\xi +p\dot{\xi} +\pderiv{L}{t}\tau - H\dot{\tau}=0,
\end{equation}
and from this identity it will be shown that 
\begin{equation}
	\label{eq:5.4}
	\lr{\xi-\dot{x}\tau}\lrs{\dot{p}-\pderiv{L}{x}} = \deriv{}{t}\lrs{p\xi-H\tau},
\end{equation}
as follows.

First it will be shown that the left-hand side of \eqref{eq:5.3} is equal to zero by expanding out the bracketed terms and substituting the derivative terms.
\begin{align*}
	\lr{\xi-\dot{x}\tau}\lrs{\dot{p}-\pderiv{L}{x}} &= \xi\dot{p} -\xi\pderiv{L}{x} - \dot{x}\tau\dot{p} + \dot{x}\tau\deriv{L}{x},\\
	&= \cancel{\xi\dot{p}} \cancel{-\xi\dot{p}} \cancel{- \dot{x}\tau\dot{p}} \cancel{+ \dot{x}\tau\dot{p}},\\
	&=0.
\end{align*}
Secondly, it will be shown that \eqref{eq:5.3} is equal to the right-hand side of \eqref{eq:5.4} which is equal to zero.
Substituting into \eqref{eq:5.3} for\[\pderiv{L}{t} = -\dot{H},\qtq{and} \dot{p} = \pderiv{L}{x}\quad\text{gives the following,}\]
\begin{align*}
	\dot{p}\xi +p\dot{\xi} &-\dot{H}\tau - H\dot{\tau}=0,\\
	\underbrace{\dot{p}\xi +p\dot{\xi}} &-\underbrace{\lr{\dot{H}\tau + H\dot{\tau}}}=0,\\
	\deriv{}{t}\lr{p\xi} &-\quad \deriv{}{t}\lr{\tau H} = 0,\\
	\therefore &\deriv{}{t}\lrs{p\xi - \tau H} = 0.
\end{align*}
Thus,\marginnote{The product rule has been used here.}[-2cm]
\[
	\lr{\xi-\dot{x}\tau}\lrs{\dot{p}-\pderiv{L}{x}} = \deriv{}{t}\lrs{p\xi - \tau H},\qtq{as required.}
\]
The differentiation of a constant is zero and also noting that zero itself is a constant, then the expression 
\[
	\deriv{}{t}\lrs{p\xi - \tau H} = 0
\]
must mean that 
\[
	p\xi - \tau H = \text{constant}.
\]