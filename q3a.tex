% !TeX t = ./TMA04.tex
To show that the following equation and boundary conditions:
\begin{equation}
\label{eq:3.1}
	\deriv{}{x}\lr{x^2\deriv{y}{x}}+\lambda x y = 0,\quad y(1)=0, y^\prime(2)=0
\end{equation}
forms a regular \sls, consider the following.

A \sls is a linear, second-order homogeneous differential equation of the form:\marginnote{see HB p28.}
\begin{equation}
\label{eq:3.2}
	\deriv{}{x}\lr{p(x)\deriv{y}{x}} + \lr{q(x) + \lambda w(x)}y=0,
\end{equation}
which is defined on a finite interval of the real axis $a<x>b$ and satisfies the following three conditions:
\begin{enumerate}
  \item 
  the functions p(x), q(x) and w(x) are real and continuous for\\ $a<x<b$;
  \item
  $p(x)$ and $w(x)$ are strictly positive for $a<x<b$;
  \item
  $p^\prime(x)$ exists and is continuous for $a\leq x\leq b$,
\end{enumerate}
together with the boundary conditions.

Comparing equations~\eqref{eq:3.1} and \eqref{eq:3.2} it is seen that $q(x)=0$ with $p(x) = x^2$ and $w(x)=x$. As $p(x)$ and $w(x)$:
\begin{enumerate}
\item
are real and continuous for $1<x<2$;
\item
are strictly positive for $1<x<2$; and
\item
$p^\prime(x)$ exists ($p^\prime(x) = 2x$) and is continuous for $1\leq x\leq 2$,
\end{enumerate}
and the two boundary conditions are given as $y(1)=0$ and $y^\prime(2)=0$, then the system is a regular \sls.

It can be shown that the system can be written as a constrained variational problem with functional
\begin{equation}
	\label{eq:3.3}
	S[y] = \Int{1}{2}{x} x^2 y^{\prime 2},\quad y(1) = 0,
\end{equation}
and constraint
\begin{equation}
	\label{eq:3.4}
	C[y] = \Int{1}{2}{x} x y^2 = 1,
\end{equation}
as follows.

For the given constrained variational problem\marginnote{$λ$ is the Lagrange multiplier.}[1cm]
\[
	\overline{F} = x^2y^{\prime2} - \lambda x y^2,
\]
and
\[
	\overline{F}_{y^\prime} = \pderiv{\overline{F}}{y^\prime}= 2x^2y^\prime\qtq{and} \overline{F}_y =\pderiv{\overline{F}}{y}= -2\lambda x y.
\]
The \el is then,
\begin{align}
\label{eq:3.5}
	&\deriv{}{x}\lr{\pderiv{\overline{F}}{y^\prime}} - \pderiv{\overline{F}}{y} = 0,\nonumber\\
	&\deriv{}{x}\lr{2x^2y^\prime} - \lr{-2\lambda x y}= 0,\nonumber\\
	&\deriv{}{x}\lr{\cancel{2}x^2\deriv{y}{x}} + \cancel{2}\lambda x y= 0,\nonumber\\
	&\deriv{}{x}\lr{x^2\deriv{y}{x}} + \lambda x y= 0,\quad y(1)=0, y^\prime(2)=0.
\end{align}
Equations \eqref{eq:3.1} and \eqref{eq:3.5} are identical showing that system \eqref{eq:3.1} can be written as a constrained variational problem.
	
